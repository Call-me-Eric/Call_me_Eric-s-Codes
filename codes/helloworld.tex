\documentclass[a4paper,12pt]{article}
\begin{document}

\title{水题选讲}
\author{Call_me_Eric}
\date{\today}
\maketitle

\section*{海亮2月2日分享}

\subsection*{CF603E}

[link](http://codeforces.com/problemset/problem/603/E)

\subsubsection*{CF603E题意}

给定一张 $n$ 个点的无向图,初始没有边。

依次加入 $m$ 条带权的边,每次加入后询问是否存在一个边集,满足每个点的度数均为奇数。

若存在,则还需要最小化边集中的最大边权。

$n \le 10^5$,$m \le 3 \times 10^5$。

----

\subsubsection*{CF603E题解}

首先想想这个条件:要求每个点的度数都是奇数。

其实这个限制已经很强了:

- 对于一个成立的图,一定不能有一个大小为奇数的联通块。
- 简单证明:每加入一条边,都会改变两个点的度数奇偶性,最开始每个点都是偶数的度数,奇数个点显然不能全都改变成奇数的度数。
- 那联通块大小是偶数就一定都成立吗?
- 一定,你先搞一个联通块的任意生成树,然后从下向上,一旦有的点度数不是奇数,那么就删除向 $fa_u$ 连接的边,这样的话,只有根有可能是不满足的。
- 但是显然是满足的,如果不满足,那么整个联通块的度数之和就是奇数,显然不可能出现这种情况。

于是问题就变成了,每次加入一条边,要求你选择一些边,使得所有的联通块都是偶数的大小,并且希望连接的边最大值最小。

不难联想到`Kruskal`最小瓶颈生成树。

想到线段树分治,从右向左维护答案。

于是问题就变成了,一条边在哪个时间段会有贡献。

我们知道,如果有一条边在某一时刻没有被选择加入最小生成树,那么任意时刻都不会再选择它了。

于是维护`Kruskal`加入边时的指针,每次更新直到满足条件或者没有边可用。

然后如果一条边被加入,那么就在 $[id,l - 1]$ 这个时间段加入这条边即可。

----

## CF1368H1&H2

[link](https://codeforces.com/contest/1368/problem/H1)

----

\subsubsection*{CF1368H1&H2题意}

实验板有 $n$ 行 $m$ 列,每个行列交叉处有一个节点。试验板每侧都有端口。左、右侧各 $n$ 个端口,上、下侧各 $m$ 个端口。每个端口是红色或蓝色。

端口可通过导线连接。

- 每根导线连接一红色端口和一蓝色端口,每个端口最多连一条导线。
- 导线的每个部分水平或垂直,最多在一个节点处拐弯。
- 导线不能在节点之外的地方和其他导线相交(也不可以和自己相交)。

试验板的容量是根据上述规则导线数量的最大值。

以下是一种可能的连接方式

![ ](https://cdn.luogu.com.cn/upload/vjudge_pic/CF1368H2/c54dfbe0d7502b7f366741129332f0a68552a265.png)

注:**Eazy Version**下**没有**修改,也就是说,$q=0$。

端口颜色未固定,有 $q$ 次修改。每次修改,在一条边上的连续区间内,端口颜色反转。

计算每次修改后试验板容量。

$1\le n,m \le 10^5, 0 \le q \le 10^5$。

----

\subsubsection*{CF1368H1题解}

首先大家看到这个东西一定能想到网络流叭!

先主动弱化数据范围,到 $n\times m\le1000$ 的时候,你会怎么做?

如果将每一个 $(i,j)$ 连向 $(i+1,j),(i,j+1)$,然后跑`Dinic`,那么最大流,也就是最大匹配,就是最终答案。

让我们回到这道题,现在 $n,m\le10^5$,显然如果对每个点都连边会T飞的,考虑怎么办。

这个时候我们发现,再考虑最大流已经很困难了,我们不妨考虑最小割。

最小割的实质其实是将整张图的所有点分割成两个点集 $S$ 和 $T$。

我们尝试将 $(x,y)\in S$ 的点 $(x,y)$ 染成黑色。

然后考虑最小割在这个图上的实际意义,其实就是所有黑白点之间边的总长。

说起来可能不太好懂,画张图更加形象。

(这张图省略了四周的颜色,实际上,黑点白点与 $R,B$ 对应)。

![ ](https://images.cnblogs.com/cnblogs_com/blogs/809443/galleries/2362603/o_231202073045_CF1368H1.png)

那么图中所有红色边长加起来就是整张图的割的代价啦(画了半天QAQ)

然后我们考虑以下几种优化(到尽可能最小)。

1. 这个图中不会出现一个完整的不与边界相连的联通块。
   - 同样的,还是上图:
   - ![ ](https://images.cnblogs.com/cnblogs_com/blogs/809443/galleries/2362603/o_231202073631_CF1368H1%20-%20%E5%89%AF%E6%9C%AC.png)
   - 发现,如果将这里面所有的黑点改成白点,最小割答案一定变小。
2. 这个图中一定不会出现一条路径,使得只沿着黑点,能够从一侧不到达相对侧
   - ![ ](https://images.cnblogs.com/cnblogs_com/blogs/809443/galleries/2362603/o_231202074322_CF1368H1%20-%20%E5%89%AF%E6%9C%AC%20(2).png)
   - 不难发现,我们将所有的黑点变成白点一定会更小。
3. 这个图中一定不会出现折线
   - ![ ](https://images.cnblogs.com/cnblogs_com/blogs/809443/galleries/2362603/o_231202074646_CF1368H1%20-%20%E5%89%AF%E6%9C%AC%20(3).png)
   - 不难发现,将这个折线“捋直了”会更短。

综上所述,如果我们想让割的代价最小,一定是**走直线找到最近的点对**。

对于行和列都是同理的,这里给出行的做法,列的做法同理。

于是可以设计 $dp_{i,0/1}$ 表示这一行的颜色是黑色还是白色的最小代价。

递推式

注:$col$ 是你之前确定好的颜色(就是不能中间改的意思)

初始状态(你强制第 $0$ 行是什么颜色,这一行的其他颜色就会产生贡献):
$$
dp_{0,0}=\sum_{i=1}^m(U_{i}=col)\\
dp_{0,1}=\sum_{i=1}^m(U_{i}\neq col)
$$
$$
dp_{i,0}=\min{dp_{i-1,0},dp_{i-1,1}+m}+(L_i=col)+(R_i=col)\\
dp_{i,1}=\min{dp_{i-1,1},dp_{i-1,0}+m}+(L_i\neq col)+(R_i\neq col)
$$
当然,如果你行匹配好了,列的就没办法匹配了,如果颜色不相同的话就得加上。
$$
dp_{0,0}=\sum_{i=1}^m(D_{i}=col)\\
dp_{0,1}=\sum_{i=1}^m(D_{i}\neq col)
$$
至此,`Eazy Version`就已经做完了。

----

\subsubsection*{CF1368H2题解}

在向下看之前请确保你已经看完并看懂了H1题解。

现在带上了修改,怎么办呢?\
同样的,在这里只说行的做法,列的做法类似。

我们发现,$(L_i,R_i)$ 的取值只有四种,递推式子也只有 $2\times2$ 这么大,于是我们不妨考虑线段树维护矩阵乘法。

在这里定义 $L_i=(ch_i==B),R_i=(ch_i==B)$
具体的,我们维护四个矩阵:
$$
\begin{Bmatrix}
2&m\\
m+2&0\\
\end{Bmatrix}^{(L_i=0,R_i=0)}
$$
$$
\begin{Bmatrix}
1&m+1\\
m+1&1\\
\end{Bmatrix}^{(L_i=1,R_i=0)}
$$
$$
\begin{Bmatrix}
1&m+1\\
m+1&1\\
\end{Bmatrix}^{(L_i=0,R_i=1)}
$$
$$
\begin{Bmatrix}
0&m+2\\
m&2\\
\end{Bmatrix}^{(L_i=1,R_i=1)}
$$
然后线段树维护下就完事了,区间翻转只需要打上相应的标记即可。

----

\end{document}